\section{Outils mis en place}

\begin{itemize}
	\item Site web en ligne (CMS Wordpress): \url{http://www.ecole.ensicaen.fr/~lvimont/} 
	\item Gestionnaire de versions (Git) 
		\begin{itemize}[label=\textbullet]
			\item Application Android: \url{https://github.com/F4r3n/Vamp} 
			\item Logiciel de traitement d'images: \url{https://github.com/F4r3n/ImageProject.git} 
		\end{itemize}
	\item Gestionnaire de projet (Trello)
\end{itemize}

\section{Technologies utilisées}

java, c++, qt, android, opencv

\section{Algorithme mis en place}

L'oeil humain est limité, en effet il est incapable de voir les subtils changements temporels. Alors
que justement en effectuant des traitements sur une vidéo. On est capable de révéler ces changements
comme par exemple, la circulation du sang. On peut même grâce à ça réussir à connaitre le pouls d'une
personne. 

Notre première idée fut simplement de réaliser une moyenne pour chaque couleur pour un pixel, puis 
une moyenne de tout les moyennes calculer par pixels. On pensait alors par la suite grâce à cette 
collection de moyenne arriver a détecter une variation. En effectuant par exemple, une dérivée puis
une amplification de la courbe obtenue.  

Dans un second temps, en restant basé sur le même principe, nous avons tenté de découper notre image
par zone de petits carrés de pixels. Par exemple une zone serait d'une taille de 5*5. On réalise 
ainsi une moyenne de chaque zone, puis par la suite une moyenne de ces moyennes. Enfin on réapplique
les mêmes fonctions qu'au-dessus.

\section{Solutions mis en place}

Durant le projet nous avons développés plusieurs outils pour répondre au besoin. Une application Android qui permettrait d'utiliser le moyen d'authentification désiré et ainsi déverouiller le smarthpone d'une
personne. Un logiciel C++ utilisant la librairie QT a également été mis en place, afin de pouvoir tester plus facilement les algorithmes que nous voulions développer. Dans la suite du projet, il a également
permis d'utiliser la webcam. 

\subsection{Développement Android}

\begin{itemize}
	\item Détection visage en temps réel
	\item Rectangle autour visage
	\item Capture de 15 images par seconde
	\item Détection couleur pixel (RGB)
	\item Caméra fonctionnelle
	\item Compteur sur 5 sec
	\item Algo mis en place  
\end{itemize}

Lockscreen

\subsection{Développement C++}

\begin{itemize}
 	\item Utilisation de avconv: résultat vidéo divisé en plusieurs images au rythme 15 FPS  
	\item Dessin d'un rect autour des images
	\item Réalisation de la moyenne
	\item Fonctions: dérivé, amplification, fft \ldots{}
	\item Filtres passe bas 
	\item Changement espaces de couleurs
	\item Capture avec une webcam (opencv)
	\item Ajout du trigger 
	\item Variations
\end{itemize}

\section{Problèmes rencontrés}

Nombres de FPS capturés avec la caméra frontal de android.\\
Limite des sessions à l'ENSICAEN, perte de travail.

