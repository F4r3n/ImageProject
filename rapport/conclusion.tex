\section{Apports techniques}

Ce projet nous a permis de mettre en œuvre l'enseignement que nous avons eu par l'ENSICAEN, l'utilisation de Github, les cours de C++ et de conception d'interface graphique. Il nous a également permis d'
apprendre le développement mobile et l'utilisation de librairie comme OpenCV\@.

\subsection{Android}
	Android est un OS sur lequel nous avions jamais programmé, l'appréhender par nous même fut vraiment difficile.\\
	Le placement des widgets sur l'écran et même le système d'activité, nous étaient inconnus et nous avions passé de nombreuses heures à le comprendre complètement.
	De plus notre projet utilisant la caméra, un des périphériques, est assez compliqué à mettre en place.
	Mais après ce projet nos connissances sur cet os se sont grandement améliorées.

\subsection{Qt}
	Nous avons déjà eu quelques expériences avec cette bibliothèque, mais nous avons réellement pu appliquer nos connaissances avec ce projet.
	Notre logiciel de test fut d'une grande aide pour l'intégration de nos algorithmes sous Android.\\
	Ce logiciel comprenant de nombreux modules, nous avons dû appliquer des méthodes vue en génie logiciel pour que celui-ci fonctionne correctement.

\subsection{OpenCV}

\section{Apports scientifiques}

Nous avons également pu mettre à profit les cours que nous avons eus sur les différents espaces de couleurs, surtout utile comme nous l'avons vu lors du développement Android. Le plus gros défi fut la mise
en place de techniques du traitement du signal, telle que la FFT ou encore la réalisation de filtre basse fréquence. 

\section{Bilan}

Ce projet, nous a forcé à aller chercher très régulièrement dans la documentation car nous découvrions une nouvelle technologie et à réutiliser tout ce que nous connaissions, ça a donc été un vrai 
aboutissement de notre formation.
