\section{Apports techniques}

Ce projet nous a permis de mettre en œuvre l'enseignement que nous a donné l'ENSICAEN, l'utilisation de Github, les cours de Java/C++ et de conception d'interface graphique. Il nous a également permis d'
apprendre le développement mobile et l'utilisation de librairie comme OpenCV\@.

\subsection{Android}
	Android est un système d'exploitation sur lequel nous n'avions jamais programmé, l'appréhender par nous même fut difficile.
	Le placement des widgets sur l'écran et même le système d'activité, nous étaient inconnus et nous avions passé de nombreuses heures à le comprendre complètement.\\
	De plus notre projet utilse la caméra, l'un des périphériques, le plus difficile à mettre en place.\\\\

	L'un des plus gros avantages d'android, c'est la documentation qui est bien écrite, ce qui est très appréciable. Grâce à ce projet, nous avons pu avoir un aperçu du système, de ces forces et
	de ces faiblesses.

\subsection{Qt}
	Nous avons déjà eu quelques expériences avec cette bibliothèque, mais nous avons réellement pu appliquer nos connaissances avec ce projet.
	Notre logiciel de test fut d'une grande aide pour l'intégration de nos algorithmes sous Android.\\
	Ce logiciel comprenant de nombreux modules, nous avons dû appliquer des méthodes vue en génie logiciel pour que celui-ci fonctionne correctement.

\subsection{OpenCV}
Nous avons dû utiliser la bibliothèque OpenCV, ce qui était une nouveauté. Celle ci nous a permis d'avoir un meilleur contrôle des images prises par la webcam.

\section{Apports scientifiques}

Nous avons également pu mettre à profit les cours que nous avons eus sur les différents espaces de couleurs, surtout utiles comme nous l'avons vu lors du développement Android.\\
Le plus gros défi fut la mise en place de techniques du traitement du signal, telle que la FFT ou encore la réalisation d'un filtre basse fréquence.


\section{Bilan}

Le diagramme de Gantt et la gestion des tâches faites au début de l'année, nous a été très utiles durant ce projet. Cela nous a permis notamment d'avoir une idée de notre avancement, afin par exemple de 
savoir si nous pouvions nous attarder sur un point précis du projet. 
Ce projet a donc été un vrai aboutissement de notre formation. Il nous a forcé à appréhender de nouvelles technologies et librairies mais également à réutiliser tout ce que nous connaissions. Nous avons pu 
voir les nombreux avantages d'une documentation bien construite et bien imagée, cela permettant de bien sur s'approprier rapidement la technologie étudiée.
