\section{Contexte}

Dans le cadre des cours de l'ENSICAEN, nous devions réaliser un projet de deuxième année. Nous avons choisi de prendre le projet reconnaissance faciale anti-fake proposé
par M. \bsc{Schwartzmann} Jean-Jacques. 

\section{Objectifs}

L'objectif du projet est la réalisation d'un moyen d'authentification par reconnaissance faciale et par différenciationn de l'homme de la photo  grâce à l'utilisation d'une capture vidéo. En effet, actuellement ce genre d'applications qui se base seulement sur la reconnaissance faciale sont facilement 
attaquables. Prenons l'exemple d'un smartphone qui se déverrouille grâce à ce moyen. Il suffit pour un attaquant de disposer de la photo de la victime pour contourner la protection. \\
Ce projet consiste donc à utiliser une capture vidéo et être capable de différencier un humain d'une photo grâce à cet enregistrement. Afin de réaliser cette différenciation, nous devions baser notre travail sur une \href{http://people.csail.mit.edu/mrub/papers/vidmag.pdf}{recherche} du MIT\@. Enfin, nous devons intégrer notre résultat dans une application Android de reconnaissance faciale fournie par le \href{https://www.greyc.fr/}{GREYC}.

\section{Contraintes}

La seule contrainte dont nous disposions était le temps de reconnaissance de la personne, en effet pour l'application Android au-dessus de 4 secondes cela était beaucoup trop long, d'un point de vue 
utilisateur. 
